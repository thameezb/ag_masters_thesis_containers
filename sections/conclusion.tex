\part{Discussion and Conclusion}

This chapter evaluates the aforementioned results in context of the posed problem.
Additionally a concluding solution is presented, with potential future-work mentioned.

\chapter{Discussion}

\section{Performance and Latency}
\section{Ease-of-Use}

\section{Cost}

\section{Resilience and Reliability}

\chapter{Conclusion}
Based on the experimentation results presented and aforementioned argued points, the following conclusions are be made
\begin{itemize}
  \item selection of container-orchestration solution will have a minor impact on performance for certain workloads
        \begin{itemize}
          \item \textit{EKS} running workloads generally performed closer to the benchmark cluster across the widest array of tests,
                with both \textit{ECS}, followed by \textit{Lambda}, lagging constantly behind, with low, but measurable performance impact.
        \end{itemize}
  \item selection of underlying server-architecture has a far more significant impact on performance for most workloads
        \begin{itemize}
          \item \textit{EC2} based workloads consistently performed closer (or on occasion out-performed) our benchmark cluster,
                whilst both \textit{Serverless} options (\textit{Fargate} and \textit{Lambda} respectively) producing lower performance,
                even under the same orchestration solution
        \end{itemize}
  \item Network latency does not mask the loss in (measurable) performance between the various orchestration solutions or underlying server-architecture
        \begin{itemize}
          \item Benchmarks showed a consistent degradation in performance between the tested solutions, irrespective of network latency.
        \end{itemize}
  \item Whilst \textit{Serverless Functions} may have the ability to run container-based workloads,
        the numerous limitations of attempting to run standard container-workloads on a
        \textit{Serverless} architectural design, makes it an inviable replacement for an existing container-orchestration solution
  \item \textit{Cloud-Native} container orchestration solutions generally offer a more \emph{consistent} experience
        (in context to the rest of the chosen \textit{cloud} provider's environment), with simpler tie-ins to other offered services, and marginally lower costs.
        These \emph{ease-of-use} factors are off-set by a measurable loss in performance, marginal (yet observable) lower level of resiliency, provider \textit{lock-in},
        and limited support.
  \item Managed Cloud \textit{Kubernetes} solutions benefit from utilizing an \textit{open-sourced}, \emph{ubiquitos} and near \textit{standard} solution
        that a \textit{k8s} cluster offers (this include that existing tooling would require little-to-no changes as part of the migration process),
        at the cost of complexity (in terms of tie-ins to other cloud offered services) and monthly costing.
\end{itemize}

\chapter{Future Work}
