\part{Methodology and System Architecture}

This chapter aims to clearly explain the research process used to analyze the problem and potential solutions.
Additionally it expands on the experimentation process undertaken in detail.

\chapter{Methodology}

\section{Research and Investigation}

A broad investigation - in the form of a literature-review was undertaken - to find, understand, compare and document the various cloud container platforms which AWS currently offers.
Additionally, interviews were performed at \emph{Allan Gray} focusing on tooling, architecture and deployment process.

\chapter{System Design and Architecture}

Each solution was evaluated in a controlled AWS \emph{VPC} in the af-south-1 region, spread across three \emph{AZs}.
The required cloud infrastructure was be deployed using \emph{Terraform} to ensure \emph{IaaC} principles are followed (and to maintain the \emph{correctness} of the testing environments).
Solutions which are not \emph{serverless} in nature were run on distinct dedicated isolated AWS EC2 instances (Virtual Machines) of type \emph{m5.2xlarge} TODO-ref, TODO-addSpec. This ensured that each solution ran independently of one another.
Solutions which are serverless in nature automatically achieved the same goal.

Benchmark results of \emph{ephemeral} objects were written and stored on AWS \emph{S3} Buckets.

\emph{MSSQL} Databases were used as part of the application benchmark process, as this is the Database Engine of choice for \emph{Allan Gray}.
Existing on-premise instances were initially used as part of the the benchmarking process, however to remove network hops as a variable factor,
a distinct AWS \emph{RDS} \emph{MSSQL} Instance was created, with a new set of benchmark runs being performed.
Each solution targeted distinct tables so as to keep the environments distinct, even at a data level.

To create a baseline, all benchmarks were performed against an existing on-premise \emph{VMWare} cloud of \emph{VMs} running \emph{RKE} TODO-addSpec (\emph{Allan Gray}'s existing container-platform) \\

\noindent Required code to recreate the testing environment can be found in TODO-addAppendix as \emph{Terraform} code.
\emph{Docker} image definitions can be found in TODO-addAppendix.
\emph{AMI} build-scripts can be found as \emph{packer} code in TODO-addAppendix.
Additionally all other tooling or applications used can be found on GitHub at TODO-addSite.

\chapter{Experimentation}
Each viable option was evaluated in terms of the following metrics:
\section{Cost}
Cost was measured in US Dollars using AWS's built in Cloud Costing tool TODO-ref, with each system matched to cost using identification tags.
Additionally, forecasted cost was predicted using AWS's Costing Calulator tool TODO-ref.

\section{Resilience and Reliability}
Resilience and Reliability were measured using \textit{TTR} when performing standard Chaos Monkey TODO-ref processes for container workloads.
These included evaluating the solution's ability to handle sudden scaling-in of container-workloads (that is to route network traffic to healthy replicas or respond to an unhealthy container by starting a new replica),
and actual deletion of container-workloads at a host level (that is stopping the container at a host-level and observing the solution's ability to recover) where possible.

\section{Performance and Latency}
Performance and Latency were measured using two forms of benchmarking, the first following industry standard performance benchmarking using the phoronix-test-suite TODO-ref tool.
With the second being a custom-built application benchmark, which was able to focus on \emph{CRUD} based activities.

\subsection{phoronix-test-suite}
The phoronix-test-suite is a high-customizable \emph{OSS} tool which acts as a wrapper to run various open-source benchmarking suites hosted by \emph{OpenBenchmarking.org} TODO-addRef.
A custom test-suite was created which targets specific aspects of a container-platform which is relevant to this project (and builds upon a previous testing methodlogy used by TODO-addRef).
Rhe selected tests are:
\begin{itemize}
  \item compress-7zip - this runs a set of 7zip compressions and decompressions against a 10GB file using \emph{7-zip}'s built in \emph{LZMA} benchmark.
        The benchmark primarily affects CPU and is measured in \emph{MIPS}.
  \item mbw - TODO
  \item ramspeed - TODO
  \item fs-mark - TODO
  \item redis - TODO
  \item m-queens - TODO
\end{itemize}
In addition to the above, a \emph{sysbench} TODO-addRef PrimeSieve benchmark was run for each solution, as an additional CPU specific benchmark.


The required tooling and configuration was then added to a Docker image built upon Ubuntu 20.04 TODO-addRef, which each solution then ran as a container.
Results were stored to \emph{S3}.

Tooling code is available on GitHub at TODO-addSite.

\subsection{tool-container-benchmark}
\textit{tool-container-benchmark} is a custom built open-source benchmarking tool. This tool performs basic \emph{CRUD} operations against a Database of choice,
which mimics a portion of a typical enterprise workload TODO-addRef. The tool starts of by clearing all projects in a specific table, creates a basic data-set within the table,
and performs a set of randomly chosen \textit{events} against a random project. An event is defined to be either a creation, update or delete operation.
The tool is written in \emph{go} TODO-addRef and was natively compiled for the chosen container \emph{OS}.

Allowed configuration:
\begin{itemize}
  \item Number of Events - this allows a user to define how many events should be run per benchmark run
  \item Number of Projects - this allows a user to define how many projects should be created per benchmark run (which creates an upper-bound of events per project)
  \item IsLambda - this configures if the benchmark should run in a server or serverless mode
\end{itemize}

Tooling code is available on GitHub at TODO-addSite.

\section{Ease-of-Use}
Ease-of-Use (being a highly subjective metric) was evaluated in terms of complexity to configure the solution environment, additional tooling required to deploy to said environment,
and finally the amount of changes required to deploy a container workload using this solution, with unit \textit{time} being the unit of comparison.
