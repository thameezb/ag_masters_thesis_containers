\begin{abstract}
  Recent years has seen mass adoption of the Containerization computing paradigm.
  In parallel, a mass adoption of the Cloud Computing paradigm has occurred. 
  Reacting to this mass adoption of Containerization, cloud-providers have rolled out various container-orchestration platforms, 
  each offering: a set of unique features, limitations, and challenges; with the divisions between each offering being fuzzy.
  This thesis compares a set of the most popular cloud container-orchestration platforms differing in of backend type (that being Virtual Machines or Serverless);
  orchestration platform (cloud-native or generic); and license (that being open-source or proprietary) in terms of:
  cost, performance, reliability and resilience, restrictions and limitations, and finally ease of adoption.
\end{abstract}