\chapter{Introduction}
\label{sec:intro}
\textit{If you can't measure it, you can't improve it --- Peter Drucker}

\section{Background}
Containerization (or OS-Level Virtualization) revolutionized the business world in 2013\cite{virtualization} with the introduction of Docker\cite{docker_usage}.
Coupled with the high adoption rate of containerization saw the rapid adoption of container-orchestration tools (with the most popular being Kubernetes\cite{k8s}).

In parallel to the adoption of containerization, cloud-computing technologies matured from basic compute, storage,
and networking services\cite{barr_2009} to more managed \textit{cloud-native} services. Amazon Web Services (AWS),
for example announced an early version of their custom container-orchestration tool (ECS) in 2015\cite{ecs},
expanding their container-orchestration offerings with a serverless container platform (Fargate) in 2017\cite{fargate},
and a managed Kubernetes Engine (EKS) in 2018\cite{eks}.
More recently, AWS has announced the ability to run containers as Lambdas (Serverless Functions) in late 2020\cite{lambda}.

\noindent These shifts in technological maturity enables workloads to fully leverage the advantages of containerization, 
with those offered by the cloud.
Unfortunately, the partition between the cloud offered container platforms are not entirely clear and rather fuzzy
(this includes performance, security, efficiency, latency and cost).\\

\noindent This Masters thesis project was conducted at Department of Information Technology at Uppsala University.
The project was undertaken in partnership with Allan Gray[\ref{sec:allan_gray}].

\subsection{Allan Gray}
\label{sec:allan_gray}
Allan Gray\cite{allan_gray} is a Africa's largest privately-owned and independent investment management company, with a primary focus on unit trusts.
Allan Gray is no exception to the aforementioned scenario. Allan Gray has been running container workloads in production for over five years,
running on the Kubernetes platform, on an on-premise data-center on virtual Linux machines.
Allan Gray is currently undertaking a cloud-migration, and is looking to shift their container-workloads to the cloud (with AWS being their chosen cloud platform).

\section{Goal}
\noindent The \textbf{goal} of this project is to investigate and compare cloud container orchestration platforms, in terms of:
\begin{itemize}
  \item ease of adoption and configuration
  \item deployment process
  \item restrictions and limitations
  \item performance
  \item cost impacts
  \item reliability and resilience
\end{itemize}

\section{Delimitations}
This project is restricted to investigation and experimentation pertaining to:
\begin{itemize}
  \item container-based workloads
  \item the Amazon Web Services Cloud provider
\end{itemize}

\subsection{Justifying the choice of focusing solely on AWS}
Whilst it seems that experimenting only on AWS severely restricts the generality of this investigation, in reality the offerings by the other two major cloud providers
(being Google Cloud Platform and Microsoft Azure Cloud respectively), match the Cloud Orchestrated Container offerings by AWS at a feature-parity level\cite{contaier_workloads}.
Additionally, AWS holds more that 33\% of the cloud market (as of Q4 2021)\cite{aws_cloud_share},
and is considered to be the cloud-provider of choice for the the largest tech companies in the world today\cite{aws_users}.

Therefore the broader results of this investigation (and recommendations) would (in theory) apply to the GCP and Azure cloud as well.

\section{Structure}
This thesis is structured as follows:
\begin{itemize}
  \item Chapter \ref{sec:theory} describes theoretical concepts, defines terminology used, and presents a brief overview of related work.
  \item Chapter \ref{sec:method} introduces the methodology used, including the system architectural design, and explains in detail the evaluation method used.
  \item Chapter \ref{sec:results} describes and presents the evaluation results.
  \item Chapter \ref{sec:discussion} discuses the results, presents concluding statements, and describes potential future work.
\end{itemize}
