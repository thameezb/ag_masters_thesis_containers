\part{Theory, Terminology, and Related Work}
\label{sec:theory}

Before analyzing the problem or providing a solution,
this chapter defines briefly the technologies and terminology used in this project.
Additionally an overview of related work is presented.

\chapter{Cloud Computing}
First coined in 1997\cite{ray2018} in an address to the INFORMS Annual meet, 
Chellapa \cite{chellappa1997intermediaries} defined Cloud Computing as a
\emph{computing paradigm where the boundaries of computing will be determined by economic rationale rather than technical limits alone}. 

Haris and Kahn\cite{haris2018systematic} provide a more expanded definition as \emph{an advancement of various combined technologies such as 
Distributed computing, Utility computing, virtualization etc, to provide IT resources and services over an internet on pay as per use manner. 
These services are available to the user on demand basis at very low cost and charged at the time of the release of resources. 
These services include storage, processing, network, application, etc}. 

\section{Cloud Computing technologies used in this project}
\subsection*{Virtual Private Cloud (VPC)}
\subsection*{Availability Zones (AZ)}
\subsection*{Elastic Compute Instance (EC2)}
\subsubsection*{Amazon Machine Image (AMI)}
\subsubsection*{Auto-Scaling Groups (ASG)}
\subsection*{Security Groups (SG)}
\subsection*{Relation Database Service(RDS)}

\chapter{Virtual Machines}

\chapter{Containerization}
\section{Docker}
\section{Kubernetes}
\section{AWS Elastic Container Service}
\section{AWS Elastic Kubernetes Service}

\chapter{Serverless Computing}
\section{Serverless Functions}
\subsection{AWS Lambda Functions}
\section{Serverless Containers}
\subsection{AWS Fargate}

\chapter{Infrastructure as Code}
\section{Terraform}
\section{packer}


\chapter{Other Terms and Concepts}
\section*{RKE}
\section*{LZMA}
\section*{CRUD}
\section*{MSSQL}
\section*{Region}
\section*{on-premise (?)}
\section*{orchestration}
\section*{burstable}
\section*{Chaos-Monkey}
\section*{Downtime}
\section*{TTR}
\section*{lock-in}
\section*{SLAs}
\section*{ephemeral}

\chapter{Related Work}
