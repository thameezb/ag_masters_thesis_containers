\part{Theory, Terminology, and Related Work}
\label{sec:theory}

Before analyzing the problem or providing a solution,
this chapter defines briefly the technologies and terminology used in this project.
Additionally an overview of related work is presented.

\chapter{Cloud Computing}
First coined in 1997\cite{ray2018} in an address to the INFORMS Annual meet, 
Chellapa \cite{chellappa1997intermediaries} defined Cloud Computing as a
\emph{computing paradigm where the boundaries of computing will be determined by economic rationale rather than technical limits alone}. 

Haris and Kahn\cite{haris2018systematic} provide a more expanded definition as \emph{an advancement of various combined technologies such as 
Distributed computing, Utility computing, virtualization etc, to provide IT resources and services over an internet on pay as per use manner. 
These services are available to the user on demand basis at very low cost and charged at the time of the release of resources. 
These services include storage, processing, network, application, etc}. 

\section{Cloud Computing technologies used in this project}

\subsection{Virtual Private Cloud (VPC)}
AWS defines a VPC as a virtual network which closely resembles a traditional network that would be operated in an \emph{on-premise} data center, 
with the benefits of using the scalable infrastructure of AWS, wherein cloud resources can be launched.\cite{awsdocs_2022}
Beach et al. further explain that a \emph{VPC allows you to configure a custom network topology, as well as manage IP routing and security. 
A network topology is the structure of the network and controls how data flows between nodes.}\cite{Beach2019}

\subsection{Regions \& Availability Zones (AZ)}
Carty defines regions as \emph{geographic locations in which public cloud service providers' data centers reside} 
and Availability Zones as \emph{isolated locations within data center regions from which public cloud services originate and operate}.\cite{carty_2015}

That is to say a public cloud service provider could have multiple isolated data-centers (zones) within a specific geographical region of operation.
Cloud resources can be deployed to various regions (and by extension specific zones) for redundancy and availability.

\subsection{Elastic Compute Cloud (EC2)}
Amazon Elastic Compute Cloud (Amazon EC2) provides scalable computing capacity in the Amazon Web Services (AWS) Cloud\cite{awsdocs_whatsisec2}.
It achieves this by allowing the creation of on-demand \emph{Virtual Machines} in the cloud called \emph{instances}\cite{carty_2019}. 
These instances can be of varying specifications \cite{daly_2022} and run various operating systems \cite{awsdocs_ec2os} per requirement.

\subsubsection{Amazon Machine Image (AMI)}
An Amazon Machine Image (AMI) is a template that contains a software configuration (for example, an operating system, an application server, and applications)\cite{awsdocs_ami}.
AMIs are published by AWS, software vendors, or can be created by users themselves to contain specific application software or configuration required per instance.
These templates are used to create EC2 instances\cite{Beach2014}. 

\subsubsection{Auto-Scaling Groups (ASG)}
AWS defines an ASG as a logical resource which \emph{contains a collection of Amazon EC2 instances that are treated as a logical grouping for the purposes of automatic scaling and management} \cite{awsdocs_asg}.
ASGs helps you ensure that you have the correct number of Amazon EC2 instances available to handle the load for your application by scaling up or down the number of active instances based on requirement \cite{amazon_asg_docs}. 
ASGs perform periodic health-checks on the instances within the group, replacing unhealthy instances with new instances to maintain availability.

\subsection{Security Groups (SG)}
AWS defines an SG as a logical resource which \emph{controls the traffic that is allowed to reach and leave the resources that it is associated with} \cite{amazon_2016}.
In essence it is a per instance set of \emph{stateful} rules which defines the allowed ingress and egress network activity per instance. 

\subsection{Relation Database Service (RDS)}
Lutkevich \cite{lutkevich_2021} defines RDS as \emph{a managed SQL database service provided by Amazon Web Services (AWS). Amazon RDS supports an array of database engines to store and organize data. 
It also helps with relational database management tasks, such as data migration, backup, recovery and patching.}

RDS supports MySQL, MariaDB, PostgreSQL, Oracle, and SQL Server \cite{beach2019relational}.

\chapter{Virtual Machines}
Defined by Popek and Goldburg in 1974 as \emph{an efficient, isolated duplicate of a real computer machine}\cite{popek_1974}. 
A VM is resource which uses software (by emulating physical hardware) to run operating systems, programs, and applications. 
Multiple isolated VMs can run as \textit{guests} on a single physical \textit{host} system, with each \textit{guest} VM reserving a portion of the underlying
\textit{host}'s hardware. 

VMs benefit from allowing easy restore points via \textit{snapshotting} (the ability to \textit{save} the entire state of a machine at any given point in time), 
and the ability to quickly create multiple VMs which share the same properties, configuration, and applications via \textit{cloning}\cite{n-able_2021}. 

Virtual Machines do have notable restrictions: infections from a \textit{host} system may spread to all \textit{guest} VMs;
the sharing and reservation of underlying hardware results in a \textit{host} system which may not be fully utilized, 
while leading to unpredictable performance for \textit{guest} VMs if the \textit{host} is under load. 
Finally due to the overhead of virtualization, the \textit{guest} VM generally performs slower than non-virtualized environments\cite{Martinovic}.

\chapter{Containerization}
\section{Docker}
\section{Kubernetes}
\section{AWS Elastic Container Service}
\section{AWS Elastic Kubernetes Service}

\chapter{Serverless Computing}
\section{Serverless Functions}
\subsection{AWS Lambda Functions}
\section{Serverless Containers}
\subsection{AWS Fargate}

\chapter{Infrastructure as Code}
\section{Terraform}
\section{packer}


\chapter{Other Terms and Concepts}
\section*{Rancher Kubernetes Engine (RKE)}
RKE is a Kubernetes distribution which runs entirely using Docker containers \cite{rke}. 

\section*{LZMA}
LZMA compression is a type of data-compression algorithm.
It was designed by Igor Pavlov as part of the 7zip \cite{pavlov_2022} project and was first implemented in 1998.
The name \textit{LZMA} stands for \textit{Lempel-Ziv Markov chain Algorithm}\cite{winzip_2021}

\section*{CRUD}
In computer programming, Create, Read, Update, and Delete are the four basic operations of persistent storage\cite{martin1983managing}.

\section*{Orchestration}
Orchestration is the automated configuration, coordination, and management of computer systems and software\cite{erl1900service}.

\section*{Chaos-Engineering}
Chaos Engineering is the discipline of experimenting on a system in order to build confidence in the system's capability to withstand turbulent conditions in production\cite{hochstein_2019}.

\section*{Time-to-recover (TTR)}
The measure of time between a service being down, until service is restored.

\section*{Service Level Agreements (SLAs)}
Service Level Agreements (SLAs) are a common way to formally specify the exact
conditions (both functional and non-functional) under which services are or should
be delivered \cite{wieder2011service}.
\chapter{Related Work}
